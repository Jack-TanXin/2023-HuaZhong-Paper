\section{参考文献}


% -------改下面的参考文献内容就行,别动这两句话!!!------- %
\renewcommand{\refname}{}
\vspace{-3em}  
% -------改下面的参考文献内容就行,别动这两句话!!!------- %


\begin{thebibliography}{200}  
	\bibitem{ref1}郭躬德,黄杰,陈黎飞. 基于KNN模型的增量学习算法[J]. 模式识别与人工智能,2010,23(05):701-707.
	\bibitem{ref2}申晴,张连增. 一种新的银行信用风险识别方法:SVM-KNN组合模型[J]. 金融监管研究,2020,(07):23-37.
	\bibitem{ref3}Benarafa Halima,Benkhalifa Mohammed,Akhloufi Moulay. WordNet Semantic Relations Based Enhancement of KNN Model for Implicit Aspect Identification in Sentiment Analysis[J]. International Journal of Computational Intelligence Systems,2023,16(1).
	\bibitem{ref4}王大鹏, 闫肃, 王楠, 等. 基于卡方检验和秩和检验的智慧消防行业分析[J]. 消防科学与技术, 2022, 41(11): 1594.
	\bibitem{ref5}王皓辰, 张长伦, 黎铭亮. 基于深度学习的点云上采样算法研究[J]. Journal of Image and Signal Processing, 2023, 12: 21.
	\bibitem{ref6}梁胜杰,张志华,崔立林. 主成分分析法与核主成分分析法在机械噪声数据降维中的应用比较[J]. 中国机械工程,2011,22(01):80-83.
	\bibitem{ref7}曹前. 基于二阶多项式回归和权重主成分分析法的多光谱降维算法研究[J]. Optical Technique, 2023, 49(2): 250-256.
	\bibitem{ref8}Wan Minghua,Wang Xichen,Tan Hai,Yang Guowei. Manifold Regularized Principal Component Analysis Method Using L2,p-Norm[J]. Mathematics,2022,10(23).
    \bibitem{ref9}张凯, 张科. 基于 LightGBM 算法的边坡稳定性预测研究[J]. 中国安全科学学报, 2022, 32(7): 113.
    \bibitem{ref10}Yang Qiang,Feng Yan,Guan Li,Wu Wenyu,Wang Sichen,Li Qiangyu. X-Band Radar Attenuation Correction Method Based on LightGBM Algorithm[J]. Remote Sensing,2023,15(3).
    \bibitem{ref11}Anand L.,Mewada Shivlal,Shamsi WameedDeyah,Ritonga Mahyudin,Aflisia Noza,KumarSarangi Prakash,NdoleArthur Moses. Diagnosis of Prostate Cancer Using GLCM Enabled KNN Technique by Analyzing MRI Images[J]. BioMed Research International,2023,2023.
    \bibitem{ref12}张晓辉, 李莹, 王华勇, 等. 应用特征聚合进行中文文本分类的改进 KNN 算法[J]. 东北大学学报: 自然科学版, 2003, 24(3): 229-232.
    \bibitem{ref13}Roshanfekr Behnam,Amirmazlaghani Maryam,Rahmati Mohammad. Learning graph from graph signals: An approach based on sensitivity analysis over a deep learning framework[J]. Knowledge-Based Systems,2023,260.
    \bibitem{ref14}彭高辉, 王志良. 数据挖掘中的数据预处理方法[J]. 华北水利水电学院学报, 2008 (6): 61-63.
    \bibitem{ref15}ParedesSalazar Enrique A,CalderónCárdenas Alfredo,Varela Hamilton. Sensitivity Analysis in the Microkinetic Description of Electrocatalytic Reactions.[J]. The journal of physical chemistry. A,2022,126(17).
    \bibitem{ref16}吴庶宸, 戚宗锋, 李建勋. 基于深度学习的智能全局灵敏度分析[J]. 上海交通大学学报, 2022, 56(7): 840.
    \bibitem{ref17}任家东,刘新倩,王倩,何海涛,赵小林. 基于KNN离群点检测和随机森林的多层入侵检测方法[J]. 计算机研究与发展,2019,56(03):566-575.
    \bibitem{ref18}Li Zhenglei,Chen Yu,Tao Yan,Zhao Xiuge,Wang Danlu,Wei Tong,Hou Yaxuan,Xu Xiaojing. Mapping the personal PM2.5 exposure of China's population using random forest[J]. Science of the Total Environment,2023,871.
    \bibitem{ref19}张著英,黄玉龙,王翰虎. 一个高效的KNN分类算法[J]. 计算机科学,2008,(03):170-172.
    \bibitem{ref20}方匡南,吴见彬,朱建平,谢邦昌. 随机森林方法研究综述[J]. 统计与信息论坛,2011,26(03):32-38.
\end{thebibliography} 