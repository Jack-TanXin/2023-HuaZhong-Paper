\begin{abstract}
    肠胃微创手术中需要使用局部的镇静和镇痛药物,现有一种新型药物“R药”有待非干预性研究。本文基于新型、传统镇静药物在临床试验中的真实表现数据,对IPI等生命体征、不良反应和患者满意度进行分析与挖掘。
    
    针对问题一:本文首先对数据进行\textbf{清洗、编码与归一化},接着基于\textbf{多变量可视化}分析得出\textbf{不同药组关于各不良反应均有显著的差异},并对不同药组关于各个不良反应分别进行\textbf{卡方检验},得出\textbf{不同药组关于术中不良反应均有显著差异,关于术后不良反应中仅“恶心呕吐”和“腹胀腹痛”有显著差异}。关于对不良反应的预判,本文对数据集进行\textbf{上采样},并基于\textbf{K最近邻}算法建立模型,经测试模型在数据集上的\textbf{分类AUC正确率均在0.92以上},并作出\textbf{混淆矩阵}和\textbf{ROC图}对模型的具体测试情况进行可视化。
    
    针对问题二:本文首先对数据集进行预处理,接着基于\textbf{主成分分析法}在\textbf{将数据降维至15维}。然后对15个主成分进行正态性检验,并分别对主成分进行\textbf{独立样本T检验}和\textbf{Mann-Whitney U检验},\textbf{不同药组仅关于第六个主成分有显著差异}。关于探究造成显著差异的原因,本文以第六主成分为标签、基于\textbf{LightGBM}建立回归模型,在模型性能优良的前提下进行\textbf{树模型的特征重要性评价},最终认为\textbf{造成显著差异的因素由大到小依次是样本的年龄、体重、体重、镇静药名称}。
    
    针对问题三:本文首先进行数据清洗以及特征的选择、缩放、编码,接着类似于问题二对数据进行降维,以避免分类特征造成数据集特征离散。然后分别以六个时间点的IPI维标签、基于\textbf{岭回归}和\textbf{支持向量机}建立回归模型,并通过\textbf{加权平均法}对二者进行融合。最后通过\textbf{MAE}和\textbf{MSE}对三个模型进行评价,并基于数据扰动对模型进行\textbf{灵敏度分析},结果显示各标签的模型在测试中的MAE和MSE基本分布在0.1到0.3,模型在经过30\%内的数据扰动后仍保持测试的MSE稳定,故知\textbf{模型具有良好的性能与稳定性}。
    
    针对问题四:本文首先基于\textbf{斯皮尔曼和肯德尔相关系数}分别对数值特征和分类特征进行相关性分析,得出镇痛药总剂量、moaasjinjing、镇静药的名称、镇静药诱导剂量、镇静药总剂量与术后满意度有相对较大的关联。关于术后满意度具体定量关系的挖掘,本文对数据集进行上采样,并基于\textbf{随机森林}建立分类模型,以\textbf{准确率、召回率、F1\_score和支持率}评价其在测试集上的性能,结果显示模型\textbf{分类准确率为0.75}。最后导出随机森林模型的\textbf{树状图},并找出模型划分类别的标准,其中涉及petco2025、镇静药总剂量等,本文\textbf{基于树状图绘制一个流程图用于表达其间的定量关系},此标准即为题设所要求的部分特征与术后满意度的联系。
    
    \vspace{1em} % 移除2个行距的空白:这句代码千万不能动!!!移除2个行距的空白:这句代码千万不能动!!!
    
    \noindent{\textbf{关键词:}主成分分析法;LightGBM;加权平均法; 灵敏度分析; 随机森林}
\end{abstract} 