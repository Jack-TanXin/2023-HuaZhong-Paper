\begin{abstract}

    摘要是全文最重要的部分,也是阅卷人首先看到的部分,阅卷人会只根据摘要将文章分成三六九等,所以如果不认真写摘要的话你就会有大麻烦,请务必留至少两小时用于摘要的打磨,将其控制在1面!!!
    
    针对问题一:在写摘要的时候请搞清楚,首先摘要的第一段是题设的背景,你需要随便胡扯几句,但别扯太多,毕竟要把摘要控制在一面很难!然后写完第一段后从第二段开始就要开始介绍你对每个题目的理解、过程以及求解与评价,语言尽量精炼,不要啰嗦,再强调一次:那么多内容的摘要压缩在一面非常难!下面我将给大家演示如何对于自己已经完成的“模型建立与求解”部分进行摘要描述。
    
    针对问题二,本文基于可视化和假设检验对所给数据集进行数据分析。首先对产品的销售价格和销售量的关系进行探究,本文分别通过\textbf{斯皮尔曼相关系数}对相关性进行定量描述,求得$\rho$=-0.2946,反映出\textbf{销售价格和销售量的相关性较弱}。接着对区域与销量的关系进行探究,通过\textbf{方差检验}得知\textbf{不同地区对订单的需求量有显著差异},并通过直方图探究出不同区域产品的需求量的不同特性。然后,本文对产品的销售方式与需求量的关系进行探究,通过\textbf{Mann-Whitney U}检验得出\textbf{线上销售与线下销售的销售量存在显著差异}。最后,通过\textbf{单样本Wilcoxon符号秩检验}将时间序列整体与促销活动单日的销量相比较,得出\textbf{促销活动单日的销量与平时的销量有显著差异}。为了将各特征的分布以及定量分析所得出的结论直观化,本文利用小提琴图、箱线图、直方图等对相关特征进行可视化,结果与定量分析一致。
    
    针对问题三,摘要内容摘要内容摘要内容摘要内容摘要内容摘要内容摘要内容摘要内容摘要内容摘要内容摘要内容摘要内容摘要内容摘要内容摘要内容摘要内容摘要内容摘要内容摘要内容摘要内容摘要内容摘要内容摘要内容摘要内容摘要内容摘要内容摘要内容摘要内容摘要内容摘要内容摘要内容摘要内容摘要内容摘要内容摘要内容摘要内容摘要内容摘要内容摘要内容摘要内容摘要内容摘要内容摘要内容摘要内容摘要内容摘要内容摘要内容摘要内容摘要内容摘要内容摘要内容摘要内容摘要内容摘要内容摘要内容摘要内容摘要内容。
    
    第二部分分别是针对数据分析的任务和机器学习的任务,可以看到摘要第一句话用于概括整个小题大致是怎么做的,从第二句话开始展开每一步,用“首先”、“接着”、“然后”、“最后”依次描述。在摘要中务必要把你的结果和评价情况直接呈现出来并且加粗,同时你使用的关键方法也需要加粗,通过这个源文件相信你已经知道该如何加粗了。
    
    \vspace{1em} % 移除2个行距的空白:这句代码千万不能动!!!移除2个行距的空白:这句代码千万不能动!!!
    
    \noindent{\textbf{关键词:}随机森林;方差选择法;Voting Classifier;层次聚类分析;决策树}
    
\end{abstract} 