\section{符号说明}

% 这里涉及到表格插入的问题,这里真的不太好描述,大家自己摸索吧,很简单。
% 然后提醒一下,如果使用表注,将会依次计入序号,如果不使用就不计入序号
% 也就是说:第一个使用表注的表是表1,而不是第一个出现的表


\begin{table}[H]
	\centering  % 不要动!
	\begin{tabular}{c c}  % 有几列就要有几个c,c与c之间用空格
		\toprule[1.5pt]  % 不要动!
		符号 & 含义  \\   % 列与列之间用  & 隔开,下同
		\midrule[1pt]    % 不要动!
		$i=1,i=2$ & 分别表示高钾、铅钡玻璃 \\ 
		$j$ & 表示表中从二氧化硅($SiO_2$)到二氧化硫($SO_2$)中第$j$类化学物质 \\
		$z=1,z=2$ & 分别表示风化前和风化后 \\
		$x_1,x_2,x_3,x_4$ & 分别表示纹饰、类型、颜色、风化表面 \\
		$y_j$ & 表示第$j$类化学物质的含量 \\ 
		$\overline{y_j}$ & 表示第$j$类化学物质的平均含量 \\  
		\toprule[1.5pt]  % 不要动!
	\end{tabular}  
    % \caption{表注:符号说明部分的表不需要表注,从下一张表开始需要表注,并记入序号}
\end{table} 