\section{符号说明}

% 这里涉及到表格插入的问题,这里真的不太好描述,大家自己摸索吧,很简单。
% 然后提醒一下,如果使用表注,将会依次计入序号,如果不使用就不计入序号
% 也就是说:第一个使用表注的表是表1,而不是第一个出现的表


\begin{table}[H]
	\centering  % 不要动!
	\begin{tabular}{c c}  % 有几列就要有几个c,c与c之间用空格
		\toprule[1.5pt]  % 不要动!
		符号 & 含义  \\   % 列与列之间用  & 隔开,下同
		\midrule[1pt]    % 不要动!
        ${{\chi }^{2}}$	            &  卡方检验的卡方值  \\
        ${{O}_{ij}}$                &  第$i$个样本的第$j$种特征存在数量  \\
        ${{E}_{ij}}$                &  第$i$个样本的第$j$种特征不存在数量  \\
        $df$	                    &  特征的自由度  \\
        $p$	                        &  样本的p值  \\
        $\chi$                      &  特征空间  \\
        ${{\overrightarrow{x}}_{i}}$ &	第$i$个样本的特征向量  \\
        ${{L}_{3}}\left( \cdot  \right)$ & 计算闵可夫斯基距离  \\
        ${{c}_{j}}$	 &  第$j$个分类指标  \\
        ${{d}_{i}}$	 &  第$i$个样本的等级  \\
        $H(\overrightarrow{x})$  &  组合模预测型向量函数  \\
        $erro{{r}_{ij}}$	     &  第$i$个特征的第$j$个样本扰动值  \\
		\toprule[1.5pt]  % 不要动!
	\end{tabular}  
\end{table} 







