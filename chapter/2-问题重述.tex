\section{问题重述}

\subsection{问题背景}

新药物研究是临床研究中的关键环节。在肠胃微创手术中,往往需要使用局部的镇静和镇痛药物,传统的镇静药物为“B药”,某药物研发中心研发了一种新型药物“R药”,新药物投入使用,通常需要经历生物试验和临床试验两个阶段。 

为了解新药物的药性特征,要研究分析病患在术中、术后的不良反应;病患术后的生命体征以及病患的满意度及其相关问题,根据实验数据的分析,对以上方面做出预估,为药物选择提供重要参考,为医师病患提供预判依据。


\subsection{问题提出}

本题附件 1 中收集了新型药物和传统镇静药物在临床试验中的表现数据,数据包括患者基本信息、术中用药、术中患者身体状态和术后患者反馈信息等,根据附件所给数据信息建立数学模型,解决一下问题:

\textbf{问题1:} 首先根据术中、术后 24小时不良反应,判断新药组和原药组是否存在显著差异;接着根据患者基本信息和镇静药物种类建立预判患者术中、术后 24小时是否会出现不良反应的数学模型。 

\textbf{问题2:} 首先判断新药组和原药组在生命体征数据方面是否表现出显著差异;若有显著差异,通过模型挖掘造成生命体征数据有显著差异的影响因素。

\textbf{问题3:} 用药后3分钟内的IPI数据在临床研究具有很大的参考意义,要求根据用药信息和患者信息预测给药后3分钟以内的 IPI 数据。

\textbf{问题4:} 要求基于现有数据找出影响术后满意度的因素,并分析影响因素与术后满意度之间的联系。