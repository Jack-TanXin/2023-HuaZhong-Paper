\section{问题分析}

本文对于赛题题设,从多个角度对数据集进行统计分析与统计推断,本部分将对四个问题进行简要分析,同时赛题的每个问题的解答完整流程以流程图的形式在附录A给出。

\subsection{问题一的分析}

针对问题一,第一问要求关于术中、术后 24h 不良反应,判断新药组和原有药物组是否存在显著差异。首先对附件1中进行数据清洗、特征缩放、特征编码,由于不良反应在编码后均为分类特征,故用基于多变量可视化分析的定性方法和基于卡方检验的定量方法探究不同药组对于不良反应的差异性。

第二问要求建立一个有效的分类模型用于对患者的不良反应进行预判。本文基于经过数据预处理的数据集,首先对标签进行分布分析,并对标签比例严重失衡的数据集进行上采样处理。接着把数据按比例分为训练集和测试集。然后基于决策树对训练集进行特征提取,使用经过训练的决策树模型实现对两大类、8组不良反应的预判。最后基于混淆矩阵、ROC图对模型在测试集上的表现进行评价。


\subsection{问题二的分析}

针对问题二,第一问要求判断新药组和原有药物组在生命体征数据方面是否表现出显著差异。首先对数据进行数据清洗和特征编码。接着基于Spearman、Kendall相关性分析计算特征之间的相关性并删除相关性高特征之一,为避免高维数据集造成数据分析困难,基于主成分分析法在保留较高方差解释比例的前提下对数据进行降维。最后对降维后的特征进行正态性检验,分别基于独立样本t检验和Mann-Whitney U研究不同药组对生命体征数据的差异性。

第二问在第一问的基础上,探究造成新药对生命体征数据显著差异的影响因素。本文基于经过数据预处理的数据集,基于LightGBM对以受试者情况和病史为特征空间、以检验出显著差异的生命特征组为标签的数据集进行特征提取,在确保模型性能极好的前提下并利用树模型独有的特征重要性评价功能对各特征对标签的作用效果进行评价,最后利用条形统计图呈现。

\subsection{问题三的分析}

针对问题三,要求根据用药信息和患者信息预测对给药后 3 分钟以内的 IPI 数据。首先对数据进行数据清洗和特征编码。接着为了避免大量分类特征造成特征空间离散从而影响模型提取特征,基于主成分分析法对特征空间进行降维。然后基于岭回归和支持向量机回归分别对数据集进行特征提取,并用线性加权法对二者结果进行融合。最后为了提高模型的可靠性和准确性,本文基于MAE、MSE和数据扰动对模型在测试集上的表现和回归的稳定性进行评价。


\subsection{问题四的分析}

针对问题四,要求基于现有数据找出与术后满意度有关的因素,本文使用定性方法和定量方法分别进行模型求解。

基于斯皮尔曼、肯德尔相关性分析分别对数值特征和分类特征与术后满意度的相关性进行探究。首先对数据进行数据清洗,对分类特征进行特征编码,对数值特征进行特征缩放。接着对两类特征分别进行相关性分析,求得相关系数。最后用条形统计图对结果进行可视化,得出初步结论。

基于树模型的树状图对术后满意度划分的定量探究。首先对数据进行预处理,并将五维关于术后满意度的特征聚合为一维关于术后满意度的评分作为标签,为避免术后满意度评分分布不均衡导致模型难以提取特征,对数据集进行上采样。然后按比例将数据集划分为训练集和测试集,并基于随机森林对训练集进行特征提取。最后导出随机森林的树状图,并通过树状图获取分类的具体依据。














